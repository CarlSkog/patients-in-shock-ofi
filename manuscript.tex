% Options for packages loaded elsewhere
\PassOptionsToPackage{unicode}{hyperref}
\PassOptionsToPackage{hyphens}{url}
%
\documentclass[
]{article}
\usepackage{amsmath,amssymb}
\usepackage{iftex}
\ifPDFTeX
  \usepackage[T1]{fontenc}
  \usepackage[utf8]{inputenc}
  \usepackage{textcomp} % provide euro and other symbols
\else % if luatex or xetex
  \usepackage{unicode-math} % this also loads fontspec
  \defaultfontfeatures{Scale=MatchLowercase}
  \defaultfontfeatures[\rmfamily]{Ligatures=TeX,Scale=1}
\fi
\usepackage{lmodern}
\ifPDFTeX\else
  % xetex/luatex font selection
\fi
% Use upquote if available, for straight quotes in verbatim environments
\IfFileExists{upquote.sty}{\usepackage{upquote}}{}
\IfFileExists{microtype.sty}{% use microtype if available
  \usepackage[]{microtype}
  \UseMicrotypeSet[protrusion]{basicmath} % disable protrusion for tt fonts
}{}
\makeatletter
\@ifundefined{KOMAClassName}{% if non-KOMA class
  \IfFileExists{parskip.sty}{%
    \usepackage{parskip}
  }{% else
    \setlength{\parindent}{0pt}
    \setlength{\parskip}{6pt plus 2pt minus 1pt}}
}{% if KOMA class
  \KOMAoptions{parskip=half}}
\makeatother
\usepackage{xcolor}
\usepackage[margin=1in]{geometry}
\usepackage{longtable,booktabs,array}
\usepackage{calc} % for calculating minipage widths
% Correct order of tables after \paragraph or \subparagraph
\usepackage{etoolbox}
\makeatletter
\patchcmd\longtable{\par}{\if@noskipsec\mbox{}\fi\par}{}{}
\makeatother
% Allow footnotes in longtable head/foot
\IfFileExists{footnotehyper.sty}{\usepackage{footnotehyper}}{\usepackage{footnote}}
\makesavenoteenv{longtable}
\usepackage{graphicx}
\makeatletter
\def\maxwidth{\ifdim\Gin@nat@width>\linewidth\linewidth\else\Gin@nat@width\fi}
\def\maxheight{\ifdim\Gin@nat@height>\textheight\textheight\else\Gin@nat@height\fi}
\makeatother
% Scale images if necessary, so that they will not overflow the page
% margins by default, and it is still possible to overwrite the defaults
% using explicit options in \includegraphics[width, height, ...]{}
\setkeys{Gin}{width=\maxwidth,height=\maxheight,keepaspectratio}
% Set default figure placement to htbp
\makeatletter
\def\fps@figure{htbp}
\makeatother
\setlength{\emergencystretch}{3em} % prevent overfull lines
\providecommand{\tightlist}{%
  \setlength{\itemsep}{0pt}\setlength{\parskip}{0pt}}
\setcounter{secnumdepth}{-\maxdimen} % remove section numbering
\newlength{\cslhangindent}
\setlength{\cslhangindent}{1.5em}
\newlength{\csllabelwidth}
\setlength{\csllabelwidth}{3em}
\newlength{\cslentryspacingunit} % times entry-spacing
\setlength{\cslentryspacingunit}{\parskip}
\newenvironment{CSLReferences}[2] % #1 hanging-ident, #2 entry spacing
 {% don't indent paragraphs
  \setlength{\parindent}{0pt}
  % turn on hanging indent if param 1 is 1
  \ifodd #1
  \let\oldpar\par
  \def\par{\hangindent=\cslhangindent\oldpar}
  \fi
  % set entry spacing
  \setlength{\parskip}{#2\cslentryspacingunit}
 }%
 {}
\usepackage{calc}
\newcommand{\CSLBlock}[1]{#1\hfill\break}
\newcommand{\CSLLeftMargin}[1]{\parbox[t]{\csllabelwidth}{#1}}
\newcommand{\CSLRightInline}[1]{\parbox[t]{\linewidth - \csllabelwidth}{#1}\break}
\newcommand{\CSLIndent}[1]{\hspace{\cslhangindent}#1}
\usepackage{booktabs}
\usepackage{longtable}
\usepackage{array}
\usepackage{multirow}
\usepackage{wrapfig}
\usepackage{float}
\usepackage{colortbl}
\usepackage{pdflscape}
\usepackage{tabu}
\usepackage{threeparttable}
\usepackage{threeparttablex}
\usepackage[normalem]{ulem}
\usepackage{makecell}
\usepackage{xcolor}
\usepackage{caption}
\usepackage{anyfontsize}
\ifLuaTeX
  \usepackage{selnolig}  % disable illegal ligatures
\fi
\IfFileExists{bookmark.sty}{\usepackage{bookmark}}{\usepackage{hyperref}}
\IfFileExists{xurl.sty}{\usepackage{xurl}}{} % add URL line breaks if available
\urlstyle{same}
\hypersetup{
  pdftitle={Opportunities for improvement in the care of adult trauma patients arriving in shock},
  pdfauthor={Carl Skog},
  hidelinks,
  pdfcreator={LaTeX via pandoc}}

\title{Opportunities for improvement in the care of adult trauma
patients arriving in shock}
\author{Carl Skog}
\date{}

\begin{document}
\maketitle

\hypertarget{abstract}{%
\section{Abstract}\label{abstract}}

\hypertarget{background}{%
\subsection{Background}\label{background}}

\hypertarget{methods}{%
\subsection{Methods}\label{methods}}

\hypertarget{results}{%
\subsection{Results}\label{results}}

\hypertarget{conclusion}{%
\subsection{Conclusion}\label{conclusion}}

\hypertarget{introduction}{%
\section{Introduction}\label{introduction}}

\hypertarget{introduction-1}{%
\section{Introduction}\label{introduction-1}}

\hypertarget{epidemiology}{%
\subsection{Epidemiology}\label{epidemiology}}

Trauma is a major global public health concern. It causes over four
million deaths, among those younger populations are the most affected
(1). Beyond the high mortality rate, trauma also imposes a substantial
socio-economic burden, due to long-term disabilities and rehabilitation
needs (1). Hemorrhagic shock, the second most common cause of death
following trauma, is the leading preventable cause of death within the
first 24 hours post-injury (2--4).

\hypertarget{shock-classifications}{%
\subsection{Shock classifications}\label{shock-classifications}}

Out of the clinical needs to assess the severity of trauma shock,
patients can be categorized by stratifying symptoms and physiological
parameters into degrees. Many different scoring systems have been
developed, such as ATLS classification for hemorrhagic shock (See Table
@ref(tab:alts) (5)), Trauma Associated Severe Hemorrhage (TASH) and
Assessment of Blood Consumption (ABC) scores for predicting massive
transfusion (6). While none of these systems comprehensively classifies
shock severity across all types of trauma shock, each aid in assessing
severity through core physiological indicators. Ultimately, these
classifications rely on vital signs and clinical indicators,
highlighting the importance of physiological metrics in evaluating shock
severity across clinical settings.

\begin{longtable}[]{@{}
  >{\raggedright\arraybackslash}p{(\columnwidth - 8\tabcolsep) * \real{0.2756}}
  >{\raggedright\arraybackslash}p{(\columnwidth - 8\tabcolsep) * \real{0.1024}}
  >{\raggedright\arraybackslash}p{(\columnwidth - 8\tabcolsep) * \real{0.1654}}
  >{\raggedright\arraybackslash}p{(\columnwidth - 8\tabcolsep) * \real{0.2126}}
  >{\raggedright\arraybackslash}p{(\columnwidth - 8\tabcolsep) * \real{0.2441}}@{}}
\caption{ATLS classification for haemorrhagic shock, adapted from (5)
\{\#tbl:atls\}.}\tabularnewline
\toprule\noalign{}
\begin{minipage}[b]{\linewidth}\raggedright
\textbf{Parameter}
\end{minipage} & \begin{minipage}[b]{\linewidth}\raggedright
\textbf{Class I}
\end{minipage} & \begin{minipage}[b]{\linewidth}\raggedright
\textbf{Class II} (Mild)
\end{minipage} & \begin{minipage}[b]{\linewidth}\raggedright
\textbf{Class III} (Moderate)
\end{minipage} & \begin{minipage}[b]{\linewidth}\raggedright
\textbf{Class IV} (Severe)
\end{minipage} \\
\midrule\noalign{}
\endfirsthead
\toprule\noalign{}
\begin{minipage}[b]{\linewidth}\raggedright
\textbf{Parameter}
\end{minipage} & \begin{minipage}[b]{\linewidth}\raggedright
\textbf{Class I}
\end{minipage} & \begin{minipage}[b]{\linewidth}\raggedright
\textbf{Class II} (Mild)
\end{minipage} & \begin{minipage}[b]{\linewidth}\raggedright
\textbf{Class III} (Moderate)
\end{minipage} & \begin{minipage}[b]{\linewidth}\raggedright
\textbf{Class IV} (Severe)
\end{minipage} \\
\midrule\noalign{}
\endhead
\bottomrule\noalign{}
\endlastfoot
Approximate blood loss & \textless15\% & 15--30\% & 31--40\% &
\textgreater40\% \\
Heart rate & ↔ & ↔/↑ & ↑ & ↑/↑↑ \\
Blood pressure & ↔ & ↔ & ↔/↓ & ↓ \\
Pulse pressure & ↔ & ↓ & ↓ & ↓ \\
Respiratory rate & ↔ & ↔ & ↔/↑ & ↑ \\
Urine output & ↔ & ↔ & ↓ & ↓↓ \\
Glasgow Coma Scale score & ↔ & ↔ & ↓ & ↓ \\
Base deficit & 0 to --2 mEq/L & --2 to --6 mEq/L & --6 to --10 mEq/L &
--10 mEq/L or less \\
Need for blood products & Monitor & Possible & Yes & Massive Transfusion
Protocol \\
\end{longtable}

\hypertarget{trauma-quality-improvement-and-opportunities-for-improvement}{%
\subsection{Trauma quality improvement and opportunities for
improvement}\label{trauma-quality-improvement-and-opportunities-for-improvement}}

The initial management of trauma patients is highly time-sensitive and
prone to error (4,7,8). Quality Improvement (QI) initiatives address
these challenges by systematically evaluating care processes and
outcomes, aiming to reduce morbidity and mortality (9). Through
structured quality measures and collaborative reviews, QI has
strengthened trauma care and improved outcomes globally (10,11).
Identifying Opportunities for Improvement (OFI) offers a broader
perspective than traditional mortality reviews by also including
non-fatal outcomes (9). Structured multidisciplinary morbidity and
mortality (M\&M) reviews are commonly used to identify OFIs. One
systemic and comprehensive way to identify OFI is using the Donabedian
framework, which asses the quality of care by the three categories of:
structure, process, and outcome (9,12).

\hypertarget{shock-and-ofi}{%
\subsection{Shock and OFI}\label{shock-and-ofi}}

As mentioned before hemorrhagic shock is the leading cause of
preventable death within the first 24 hours of injury (3). Patients in
shock have a higher mortality rate than those not (13), with a median
time to death of just 2 hours (14). Despite this, the relationship
between shock, its severity, and Opportunities for Improvement (OFI)
remains largely unexamined. This knowledge would help clinicians better
identify and avoid preventable mistakes in shock patients who need
urgent care.

\hypertarget{aims}{%
\subsection{Aims}\label{aims}}

This study aims to describe the types of opportunities for improvement
for adult trauma patients arriving in shock, and to assess how the
degree of shock is associated with opportunities for improvement.

\hypertarget{methods-1}{%
\section{Methods}\label{methods-1}}

\hypertarget{study-design}{%
\subsection{Study design}\label{study-design}}

We conducted a registry based retrospective cohort study, using data
from the trauma registry and trauma care quality database at the
Karolinska University Hospital in Solna.

\hypertarget{setting}{%
\subsection{Setting}\label{setting}}

The trauma registry includes patients treated at Karolinska University
Hospital in Solna, which treats all major trauma in the greater
metropolitan area of Stockholm. We included patients registered between
2014 and 2023. The trauma care quality database is a subset of the
trauma registry and includes patients selected for review.

\hypertarget{participants}{%
\subsection{Participants}\label{participants}}

Inclusion in the trauma registry requires either admission through
trauma team activation or presenting with an Injury Severity Score (ISS)
greater than nine after admission to the Karolinska University Hopsital.
Each trauma patient is then included in a morbidity and mortality review
process, which involves both individual case evaluations by specialized
nurses and audit filters. Patients identified with a high potential for
OFIs are discussed at multidisciplinary conferences. The identified OFIs
are then categorized into broad and detailed categories. The presence or
absence of OFIs is determined by consensus among all participants and is
documented in the trauma care quality database.

We included all patients in the trauma registry and Trauma care quality
database. We excluded patients younger than 15 and/or were dead on
arrival.

\hypertarget{variables-and-data-sourcesmeasurements}{%
\subsection{Variables and data
sources/measurements}\label{variables-and-data-sourcesmeasurements}}

The outcome was defined as the presence of at least one OFI, determined
by the multidisciplinary M\&M conference in the trauma care quality
registry. An OFI can be various types of preventable events, and can be
categorized as: clinical judgment error, inadequate resources, delay in
treatment, missed injury, inadequate protocols, preventable death and
other errors.

The patients will be classified to different degrees of shock in
parallel and separately, once with systolic blood pressure (SBP) and
once with base excess (BE), regardless of their cause of shock. Based on
these two parameters, the patients will be classified into four classes
roughly based on the ATLS trauma shock classifications.

We included five other metrics to be adjusted for, due to potential for
confounding. Pre-injury ASA and gender were categorical, meanwhile age,
ISS and INR were kept continuous.

\hypertarget{bias}{%
\subsection{Bias}\label{bias}}

\hypertarget{study-size}{%
\subsection{Study size}\label{study-size}}

All available data in trauma registry and trauma care quality database
will be included.

\hypertarget{quantitative-variables}{%
\subsection{Quantitative variables}\label{quantitative-variables}}

The BE parameter was divided into four classes with these cutoffs: Class
I (above -2), Class II (-2 to -6), Class III (-6 to -10), and Class IV
(below -10). Similarly, SBP was divided into four classes: Class I
(above 110 mmHg), Class II (109-100 mmHg), Class III (99-90 mmHg), and
Class IV (below 90 mmHg). This classification follows the tenth edition
ATLS hemorrhage classification (5), which does not specify exact values
for any parameter other than BE. To solve this, we assigned numerical
SBP values to each class based on findings from Eastridge et al.~and
Oyetunji et al., who redefined hypotension as 110 mmHg respective 90
mmHg dependent on age(15,16). Therefore, we used 110 mmHg as the lower
limit for (Class I: no shock) - (ATLS: normal SBP) and below 90 mmHg for
(Class IV: clear shock) - (ATLS: clear hypotension), class 2 and 3
divided equally in between. This also aligns with the classification
done by Mutschler et al.(17).

\textbf{ATLS tenth edition - Systolic blood pressure and Base excess
(BD)}

\begin{longtable}[]{@{}
  >{\raggedright\arraybackslash}p{(\columnwidth - 8\tabcolsep) * \real{0.2734}}
  >{\raggedright\arraybackslash}p{(\columnwidth - 8\tabcolsep) * \real{0.1016}}
  >{\raggedright\arraybackslash}p{(\columnwidth - 8\tabcolsep) * \real{0.1719}}
  >{\raggedright\arraybackslash}p{(\columnwidth - 8\tabcolsep) * \real{0.2109}}
  >{\raggedright\arraybackslash}p{(\columnwidth - 8\tabcolsep) * \real{0.2422}}@{}}
\toprule\noalign{}
\begin{minipage}[b]{\linewidth}\raggedright
\textbf{Parameter - original ATLS 10th edition}
\end{minipage} & \begin{minipage}[b]{\linewidth}\raggedright
\textbf{Class I}
\end{minipage} & \begin{minipage}[b]{\linewidth}\raggedright
\textbf{Class II} (Mild)
\end{minipage} & \begin{minipage}[b]{\linewidth}\raggedright
\textbf{Class III} (Moderate)
\end{minipage} & \begin{minipage}[b]{\linewidth}\raggedright
\textbf{Class IV} (Severe)
\end{minipage} \\
\midrule\noalign{}
\endhead
\bottomrule\noalign{}
\endlastfoot
Base deficit & 0 to --2 mEq/L & --2 to --6 mEq/L & --6 to --10 mEq/L &
--10 mEq/L or less \\
Systolic Blood pressure & ↔ & ↔ & ↔/↓ & ↓ \\
\textbf{Parameter - numerical approximated SBP} & & & & \\
Systolic Blood pressure & \textgreater110 mmhg & 109-100 mmhg & 99-90
mmhg & \textless90 mmhg \\
\end{longtable}

Please note that ATLS classification defines class I for the BE
parameter as (0 to -2) however, we choose to follow its original
source(18), which defines class I as base deficit (Inverted BE) ≤ 2,
with no lower limit (i.e.~an upper limit for BE).

\hypertarget{statistical-methods}{%
\subsection{Statistical methods}\label{statistical-methods}}

The statistical analysis will be performed using R, a programming
language and environment for statistical computing. We will present the
types of OFI as percentage distributions and then visualize them in a
bar chart. Unadjusted logistic regression will be used to determine the
association between the OFI, and degree of shock defined separately by
classifying BE and SBP roughly according to ATLS classification of
hemorrhagic shock. Adjusted logistic regression will then incorporate
other patient factors such as age, sex, preinjury ASA, INR, and ISS. It
will be presented as odds ratios (OR) between the presence of OFI, and
shock classes. The OR will be determined with 95\% confidence intervals,
and a significance level of 5\% will be used. All statistical analysis
will first be done on synthetic data and later implemented on the data
collected from the trauma registry and the trauma care quality database
to ensure objectivity. Missing data will be addressed by listwise
deletion.

\hypertarget{results-1}{%
\section{Results}\label{results-1}}

\hypertarget{participants-1}{%
\subsection{Participants}\label{participants-1}}

There were a total of 14022 patients in the trauma registry. After
excluding the patients younger than 15 and/or were dead on arrival,
there were 12153 patients left. Out of those, 7152 patients had been
reviewed for the presence of OFI. A total of 2233 patients were excluded
due to missing data, resulting in 4919 patients for the final analysis.

The variable with most missing data is BE which lack the data for 1721
patients.

Table 3 - Sample characteristics showing missing data.

\begin{table}[!t]
\fontsize{12.0pt}{14.4pt}\selectfont
\begin{tabular*}{\linewidth}{@{\extracolsep{\fill}}lcccc}
\toprule
\textbf{Characteristic} & \textbf{Overall}  N = 7,152\textsuperscript{\textit{1}} & \textbf{No}  N = 6,716\textsuperscript{\textit{1}} & \textbf{Yes}  N = 436\textsuperscript{\textit{1}} & \textbf{p-value}\textsuperscript{\textit{2}} \\ 
\midrule\addlinespace[2.5pt]
Age (Years) & 40 (26, 56) & 40 (26, 56) & 45 (28, 61) & <0.001 \\ 
Gender (M/F) &  &  &  & 0.14 \\ 
    Female & 2,159 (30\%) & 2,041 (30\%) & 118 (27\%) &  \\ 
    Male & 4,993 (70\%) & 4,675 (70\%) & 318 (73\%) &  \\ 
Pre-injury ASA &  &  &  & 0.031 \\ 
    1 & 4,362 (61\%) & 4,121 (61\%) & 241 (55\%) &  \\ 
    2 & 1,803 (25\%) & 1,684 (25\%) & 119 (27\%) &  \\ 
    3 & 949 (13\%) & 874 (13\%) & 75 (17\%) &  \\ 
    4 & 29 (0.4\%) & 28 (0.4\%) & 1 (0.2\%) &  \\ 
    Unknown & 9 & 9 & 0 &  \\ 
Systolic blood pressure (mmhg) & 135 (121, 150) & 135 (121, 150) & 135 (120, 150) & 0.6 \\ 
    Unknown & 123 & 114 & 9 &  \\ 
Injury Severity Score & 9 (1, 16) & 8 (1, 14) & 17 (10, 24) & <0.001 \\ 
    Unknown & 2 & 2 & 0 &  \\ 
OFI categories broad &  &  &  &  \\ 
    Clinical judgement error & 157 (36\%) & 0 (NA\%) & 157 (36\%) &  \\ 
    Delay in treatment & 67 (15\%) & 0 (NA\%) & 67 (15\%) &  \\ 
    Documentation Issues & 14 (3.2\%) & 0 (NA\%) & 14 (3.2\%) &  \\ 
    Inadequate protocols & 21 (4.8\%) & 0 (NA\%) & 21 (4.8\%) &  \\ 
    Inadequate resources & 104 (24\%) & 0 (NA\%) & 104 (24\%) &  \\ 
    Missed diagnosis & 67 (15\%) & 0 (NA\%) & 67 (15\%) &  \\ 
    Other errors & 5 (1.1\%) & 0 (NA\%) & 5 (1.1\%) &  \\ 
    Unknown & 6,717 & 6,716 & 1 &  \\ 
Base Excess (BE) & 0.8 (-1.3, 2.1) & 0.9 (-1.2, 2.2) & 0.0 (-2.4, 1.5) & <0.001 \\ 
    Unknown & 1,721 & 1,597 & 124 &  \\ 
INR & 1.00 (1.00, 1.10) & 1.00 (1.00, 1.10) & 1.00 (1.00, 1.10) & <0.001 \\ 
    Unknown & 1,620 & 1,517 & 103 &  \\ 
Shock classification - BE &  &  &  & 0.003 \\ 
    Class 1 & 4,391 (81\%) & 4,162 (81\%) & 229 (73\%) &  \\ 
    Class 2 & 719 (13\%) & 664 (13\%) & 55 (18\%) &  \\ 
    Class 3 & 191 (3.5\%) & 172 (3.4\%) & 19 (6.1\%) &  \\ 
    Class 4 & 130 (2.4\%) & 121 (2.4\%) & 9 (2.9\%) &  \\ 
    Unknown & 1,721 & 1,597 & 124 &  \\ 
Shock classification - SBP &  &  &  & 0.006 \\ 
    Class 1 & 6,356 (90\%) & 5,983 (91\%) & 373 (87\%) &  \\ 
    Class 2 & 355 (5.1\%) & 331 (5.0\%) & 24 (5.6\%) &  \\ 
    Class 3 & 147 (2.1\%) & 138 (2.1\%) & 9 (2.1\%) &  \\ 
    Class 4 & 171 (2.4\%) & 150 (2.3\%) & 21 (4.9\%) &  \\ 
    Unknown & 123 & 114 & 9 &  \\ 
\bottomrule
\end{tabular*}
\begin{minipage}{\linewidth}
\textsuperscript{\textit{1}}Median (Q1, Q3); n (\%)\\
\textsuperscript{\textit{2}}Wilcoxon rank sum test; Pearson's Chi-squared test; Fisher's exact test\\
\end{minipage}
\end{table}

\hypertarget{descriptive-data}{%
\subsection{Descriptive data}\label{descriptive-data}}

The study demographics median age is 39 (25, 56) and the most common
gender is male, 3,506 (71\%). Clinically, the median ISS is 5 which is
considered minor injuries, median INR is 1,00, and the most common
Preinjury ASA class is 1, at 62\%.

Table 4 - Sample characteristics used in regression.

\begin{table}[!t]
\fontsize{12.0pt}{14.4pt}\selectfont
\begin{tabular*}{\linewidth}{@{\extracolsep{\fill}}lcccc}
\toprule
\textbf{Characteristic} & \textbf{Overall}  N = 4,919\textsuperscript{\textit{1}} & \textbf{No}  N = 4,638\textsuperscript{\textit{1}} & \textbf{Yes}  N = 281\textsuperscript{\textit{1}} & \textbf{p-value}\textsuperscript{\textit{2}} \\ 
\midrule\addlinespace[2.5pt]
Age (Years) & 39 (25, 56) & 39 (25, 55) & 44 (26, 60) & 0.016 \\ 
Gender (M/F) &  &  &  & 0.2 \\ 
    Female & 1,413 (29\%) & 1,341 (29\%) & 72 (26\%) &  \\ 
    Male & 3,506 (71\%) & 3,297 (71\%) & 209 (74\%) &  \\ 
Pre-injury ASA &  &  &  & 0.12 \\ 
    1 & 3,032 (62\%) & 2,874 (62\%) & 158 (56\%) &  \\ 
    2 & 1,240 (25\%) & 1,163 (25\%) & 77 (27\%) &  \\ 
    3 & 626 (13\%) & 580 (13\%) & 46 (16\%) &  \\ 
    4 & 21 (0.4\%) & 21 (0.5\%) & 0 (0\%) &  \\ 
Systolic blood pressure (mmhg) & 136 (122, 150) & 136 (122, 150) & 136 (120, 150) & 0.8 \\ 
Injury Severity Score & 5 (1, 14) & 5 (1, 13) & 17 (9, 22) & <0.001 \\ 
OFI categories broad &  &  &  &  \\ 
    Clinical judgement error & 103 (37\%) & 0 (NA\%) & 103 (37\%) &  \\ 
    Delay in treatment & 49 (17\%) & 0 (NA\%) & 49 (17\%) &  \\ 
    Documentation Issues & 9 (3.2\%) & 0 (NA\%) & 9 (3.2\%) &  \\ 
    Inadequate protocols & 6 (2.1\%) & 0 (NA\%) & 6 (2.1\%) &  \\ 
    Inadequate resources & 61 (22\%) & 0 (NA\%) & 61 (22\%) &  \\ 
    Missed diagnosis & 48 (17\%) & 0 (NA\%) & 48 (17\%) &  \\ 
    Other errors & 5 (1.8\%) & 0 (NA\%) & 5 (1.8\%) &  \\ 
    Unknown & 4,638 & 4,638 & 0 &  \\ 
Base Excess (BE) & 0.9 (-1.2, 2.2) & 0.9 (-1.1, 2.3) & 0.0 (-2.4, 1.4) & <0.001 \\ 
INR & 1.00 (1.00, 1.10) & 1.00 (1.00, 1.10) & 1.00 (1.00, 1.10) & <0.001 \\ 
Shock classification - BE &  &  &  & 0.004 \\ 
    Class 1 & 3,989 (81\%) & 3,783 (82\%) & 206 (73\%) &  \\ 
    Class 2 & 644 (13\%) & 592 (13\%) & 52 (19\%) &  \\ 
    Class 3 & 170 (3.5\%) & 154 (3.3\%) & 16 (5.7\%) &  \\ 
    Class 4 & 116 (2.4\%) & 109 (2.4\%) & 7 (2.5\%) &  \\ 
Shock classification - SBP &  &  &  & 0.028 \\ 
    Class 1 & 4,455 (91\%) & 4,208 (91\%) & 247 (88\%) &  \\ 
    Class 2 & 239 (4.9\%) & 225 (4.9\%) & 14 (5.0\%) &  \\ 
    Class 3 & 98 (2.0\%) & 93 (2.0\%) & 5 (1.8\%) &  \\ 
    Class 4 & 127 (2.6\%) & 112 (2.4\%) & 15 (5.3\%) &  \\ 
\bottomrule
\end{tabular*}
\begin{minipage}{\linewidth}
\textsuperscript{\textit{1}}Median (Q1, Q3); n (\%)\\
\textsuperscript{\textit{2}}Wilcoxon rank sum test; Pearson's Chi-squared test; Fisher's exact test\\
\end{minipage}
\end{table}

\hypertarget{outcome-data}{%
\subsection{Outcome data}\label{outcome-data}}

Out of the 4919 patients reviewed without missing data, 281 (5.7\%)
patients had at least one positive OFI.

The most common broad sub-ofi category was clinical judgement error, at
103 (37\%). The other three major broad sub-ofis are: inadequate
resources 61 (22\%), delay in treatment 49 (17\%) and missed diagnosis
48 (17\%). The smallest one, is other errors, at 5 (1,8\%).

When classified, the most common broad sub-ofi in class 1 (no shock) is
clinical judgement error both in the BE and SBP groups, while the other
sub-ofi also follow a similar distribution between the groups. The
largest sub-ofi continuous to be clinical judgment error for BE group in
class 2, 3 and 4, at 34,6\%; 31,2\%; and 42,9\% in this order. On other
hand, the largest sub-ofi in SBP class 2 is delay in treatment at 50\%.
The SBP class 3 is evenly distributed between clinical judgement error
and missed diagnosis at 40\% each. Lastly, class 4 SBP's most common
sub-ofi is clinical judgment error 46,7\%.

Overall 3989 (81\%) of the patients were classified as having no shock
(class 1) according to the BE classification, the sum of the remaining
classes accounted for 930 (19\%) patients. The distribution of the
classes were decreasing in numbers with increased severity, with class
2: 644 (13\%), class 3: 170 (3,5\%), and class 4: 116 (2,4\%).

The SBP classification system sorted 4455 (91\%) patients as having no
shock (class 1), while the remaining classes with patients in shock
accounted for 464 (9\%) of all the patients. Among the classes defined
with shock, class 2 was the biggest at 239 (4,9\%), followed by class 4
at 127 (2,6\%), and lastly class 3 with 99 (2,0\%).

Table 5 - Sub-OFI, classified according to BE

\begin{table}[!t]
\fontsize{12.0pt}{14.4pt}\selectfont
\begin{tabular*}{\linewidth}{@{\extracolsep{\fill}}lcccc}
\toprule
\textbf{Characteristic} & \textbf{Class 1}  N = 206\textsuperscript{\textit{1}} & \textbf{Class 2}  N = 52\textsuperscript{\textit{1}} & \textbf{Class 3}  N = 16\textsuperscript{\textit{1}} & \textbf{Class 4}  N = 7\textsuperscript{\textit{1}} \\ 
\midrule\addlinespace[2.5pt]
OFI categories broad &  &  &  &  \\ 
    Clinical judgement error & 77 (37\%) & 18 (35\%) & 5 (31\%) & 3 (43\%) \\ 
    Delay in treatment & 32 (16\%) & 12 (23\%) & 4 (25\%) & 1 (14\%) \\ 
    Documentation Issues & 6 (2.9\%) & 2 (3.8\%) & 1 (6.3\%) & 0 (0\%) \\ 
    Inadequate protocols & 5 (2.4\%) & 1 (1.9\%) & 0 (0\%) & 0 (0\%) \\ 
    Inadequate resources & 46 (22\%) & 9 (17\%) & 4 (25\%) & 2 (29\%) \\ 
    Missed diagnosis & 39 (19\%) & 8 (15\%) & 1 (6.3\%) & 0 (0\%) \\ 
    Other errors & 1 (0.5\%) & 2 (3.8\%) & 1 (6.3\%) & 1 (14\%) \\ 
\bottomrule
\end{tabular*}
\begin{minipage}{\linewidth}
\textsuperscript{\textit{1}}n (\%)\\
\end{minipage}
\end{table}

Table 6 - Sub-OFI, classified according to SBP

\begin{table}[!t]
\fontsize{12.0pt}{14.4pt}\selectfont
\begin{tabular*}{\linewidth}{@{\extracolsep{\fill}}lcccc}
\toprule
\textbf{Characteristic} & \textbf{Class 1}  N = 247\textsuperscript{\textit{1}} & \textbf{Class 2}  N = 14\textsuperscript{\textit{1}} & \textbf{Class 3}  N = 5\textsuperscript{\textit{1}} & \textbf{Class 4}  N = 15\textsuperscript{\textit{1}} \\ 
\midrule\addlinespace[2.5pt]
OFI categories broad &  &  &  &  \\ 
    Clinical judgement error & 91 (37\%) & 3 (21\%) & 2 (40\%) & 7 (47\%) \\ 
    Delay in treatment & 39 (16\%) & 7 (50\%) & 1 (20\%) & 2 (13\%) \\ 
    Documentation Issues & 8 (3.2\%) & 1 (7.1\%) & 0 (0\%) & 0 (0\%) \\ 
    Inadequate protocols & 4 (1.6\%) & 1 (7.1\%) & 0 (0\%) & 1 (6.7\%) \\ 
    Inadequate resources & 57 (23\%) & 1 (7.1\%) & 0 (0\%) & 3 (20\%) \\ 
    Missed diagnosis & 44 (18\%) & 1 (7.1\%) & 2 (40\%) & 1 (6.7\%) \\ 
    Other errors & 4 (1.6\%) & 0 (0\%) & 0 (0\%) & 1 (6.7\%) \\ 
\bottomrule
\end{tabular*}
\begin{minipage}{\linewidth}
\textsuperscript{\textit{1}}n (\%)\\
\end{minipage}
\end{table}

\hypertarget{main-results}{%
\subsection{Main results}\label{main-results}}

The adjusted analyses showed only statistical significance when
comparing BE class 4 to class 1, which had 63\% lower odds for OFI (OR
0,37; 95\% CI 0,14-0,83; p-value 0.026).

The un-adjusted analysis showed when comparing to class 1 in BE group,
class 2 (OR 1,61; 95\% CI 1,17-2,20; p-value 0.003) and 3 (OR 1,91; 95\%
CI 1,08-3,16; p-value 0.018) were significant in increased risk for OFI.
For the SBP group, comparison between class 1 and class 4 showed
significant increased risk (OR 2,28; 95\% CI 1,26-3,85; p-value 0.004).

Table 7 and 8 adjusted log

\begin{table}[!t]
\fontsize{12.0pt}{14.4pt}\selectfont
\begin{tabular*}{\linewidth}{@{\extracolsep{\fill}}lccccccc}
\toprule
 & \multicolumn{4}{c}{\textbf{Adjusted Model}} & \multicolumn{3}{c}{\textbf{Unadjusted Model}} \\ 
\cmidrule(lr){2-5} \cmidrule(lr){6-8}
\textbf{Characteristic} & \textbf{Event Risk} & \textbf{OR}\textsuperscript{\textit{1}} & \textbf{95\% CI}\textsuperscript{\textit{1}} & \textbf{p-value} & \textbf{OR}\textsuperscript{\textit{1}} & \textbf{95\% CI}\textsuperscript{\textit{1}} & \textbf{p-value} \\ 
\midrule\addlinespace[2.5pt]
Shock classification - BE &  &  &  &  &  &  &  \\ 
    Class 1 & 206 / 3,989 (5.2\%) & — & — &  & — & — &  \\ 
    Class 2 & 52 / 644 (8.1\%) & 1.12 & 0.79, 1.55 & 0.5 & 1.61 & 1.17, 2.20 & {\bfseries 0.003} \\ 
    Class 3 & 16 / 170 (9.4\%) & 0.93 & 0.50, 1.62 & 0.8 & 1.91 & 1.08, 3.16 & {\bfseries 0.018} \\ 
    Class 4 & 7 / 116 (6.0\%) & 0.37 & 0.14, 0.83 & {\bfseries 0.026} & 1.18 & 0.49, 2.39 & 0.7 \\ 
Age (Years) & 281 / 4,919 (5.7\%) & 1.01 & 1.00, 1.01 & 0.10 &  &  &  \\ 
Gender (M/F) &  &  &  &  &  &  &  \\ 
    Female & 72 / 1,413 (5.1\%) & — & — &  &  &  &  \\ 
    Male & 209 / 3,506 (6.0\%) & 1.10 & 0.84, 1.48 & 0.5 &  &  &  \\ 
Pre-injury ASA &  &  &  &  &  &  &  \\ 
    1 & 158 / 3,032 (5.2\%) & — & — &  &  &  &  \\ 
    2 & 77 / 1,240 (6.2\%) & 1.05 & 0.77, 1.41 & 0.8 &  &  &  \\ 
    3 & 46 / 626 (7.3\%) & 1.19 & 0.79, 1.76 & 0.4 &  &  &  \\ 
    4 & 0 / 21 (0.00\%) & 0.00 &  & >0.9 &  &  &  \\ 
INR & 281 / 4,919 (5.7\%) & 1.15 & 0.78, 1.52 & 0.4 &  &  &  \\ 
Injury Severity Score & 281 / 4,919 (5.7\%) & 1.06 & 1.05, 1.07 & {\bfseries <0.001} &  &  &  \\ 
\bottomrule
\end{tabular*}
\begin{minipage}{\linewidth}
\textsuperscript{\textit{1}}OR = Odds Ratio, CI = Confidence Interval\\
\end{minipage}
\end{table}
\begin{table}[!t]
\fontsize{12.0pt}{14.4pt}\selectfont
\begin{tabular*}{\linewidth}{@{\extracolsep{\fill}}lccccccc}
\toprule
 & \multicolumn{4}{c}{\textbf{Adjusted Model}} & \multicolumn{3}{c}{\textbf{Unadjusted Model}} \\ 
\cmidrule(lr){2-5} \cmidrule(lr){6-8}
\textbf{Characteristic} & \textbf{Event Risk} & \textbf{OR}\textsuperscript{\textit{1}} & \textbf{95\% CI}\textsuperscript{\textit{1}} & \textbf{p-value} & \textbf{OR}\textsuperscript{\textit{1}} & \textbf{95\% CI}\textsuperscript{\textit{1}} & \textbf{p-value} \\ 
\midrule\addlinespace[2.5pt]
Shock classification - SBP &  &  &  &  &  &  &  \\ 
    Class 1 & 247 / 4,455 (5.5\%) & — & — &  & — & — &  \\ 
    Class 2 & 14 / 239 (5.9\%) & 0.96 & 0.52, 1.64 & 0.9 & 1.06 & 0.58, 1.78 & 0.8 \\ 
    Class 3 & 5 / 98 (5.1\%) & 0.47 & 0.16, 1.13 & 0.13 & 0.92 & 0.32, 2.05 & 0.8 \\ 
    Class 4 & 15 / 127 (12\%) & 0.73 & 0.37, 1.35 & 0.3 & 2.28 & 1.26, 3.85 & {\bfseries 0.004} \\ 
Age (Years) & 281 / 4,919 (5.7\%) & 1.01 & 1.00, 1.01 & 0.076 &  &  &  \\ 
Gender (M/F) &  &  &  &  &  &  &  \\ 
    Female & 72 / 1,413 (5.1\%) & — & — &  &  &  &  \\ 
    Male & 209 / 3,506 (6.0\%) & 1.13 & 0.85, 1.51 & 0.4 &  &  &  \\ 
Pre-injury ASA &  &  &  &  &  &  &  \\ 
    1 & 158 / 3,032 (5.2\%) & — & — &  &  &  &  \\ 
    2 & 77 / 1,240 (6.2\%) & 1.03 & 0.76, 1.39 & 0.9 &  &  &  \\ 
    3 & 46 / 626 (7.3\%) & 1.17 & 0.77, 1.73 & 0.5 &  &  &  \\ 
    4 & 0 / 21 (0.00\%) & 0.00 &  & >0.9 &  &  &  \\ 
INR & 281 / 4,919 (5.7\%) & 1.14 & 0.76, 1.52 & 0.4 &  &  &  \\ 
Injury Severity Score & 281 / 4,919 (5.7\%) & 1.06 & 1.05, 1.07 & {\bfseries <0.001} &  &  &  \\ 
\bottomrule
\end{tabular*}
\begin{minipage}{\linewidth}
\textsuperscript{\textit{1}}OR = Odds Ratio, CI = Confidence Interval\\
\end{minipage}
\end{table}

\hypertarget{discussion}{%
\section{Discussion}\label{discussion}}

\hypertarget{conclusion-1}{%
\section{Conclusion}\label{conclusion-1}}

\hypertarget{references}{%
\section{References}\label{references}}

\hypertarget{refs}{}
\begin{CSLReferences}{0}{0}
\leavevmode\vadjust pre{\hypertarget{ref-noauthor_injuries_nodate}{}}%
\CSLLeftMargin{1. }%
\CSLRightInline{Injuries and violence {[}Internet{]}. {[}cited 2024 Sep
10{]}. Available from:
\url{https://www.who.int/news-room/fact-sheets/detail/injuries-and-violence}}

\leavevmode\vadjust pre{\hypertarget{ref-kauvar_impact_2006}{}}%
\CSLLeftMargin{2. }%
\CSLRightInline{Kauvar DS, Lefering R, Wade CE. Impact of {Hemorrhage}
on {Trauma} {Outcome}: {An} {Overview} of {Epidemiology}, {Clinical}
{Presentations}, and {Therapeutic} {Considerations}. Journal of Trauma
and Acute Care Surgery {[}Internet{]}. 2006 Jun {[}cited 2024 Sep
19{]};60(6):S3. Available from:
\url{https://journals.lww.com/jtrauma/fulltext/2006/06001/impact_of_hemorrhage_on_trauma_outcome__an.2.aspx}}

\leavevmode\vadjust pre{\hypertarget{ref-berry_shock_2015}{}}%
\CSLLeftMargin{3. }%
\CSLRightInline{Berry S. Shock {Management} in {Trauma}. In: Papadakos
PJ, Gestring ML, editors. Encyclopedia of {Trauma} {Care}
{[}Internet{]}. Berlin, Heidelberg: Springer; 2015 {[}cited 2024 Sep
20{]}. p. 1484--8. Available from:
\url{https://doi.org/10.1007/978-3-642-29613-0_505}}

\leavevmode\vadjust pre{\hypertarget{ref-teixeira_preventable_2007}{}}%
\CSLLeftMargin{4. }%
\CSLRightInline{Teixeira PGR, Inaba K, Hadjizacharia P, Brown C, Salim
A, Rhee P, et al.
\href{https://doi.org/10.1097/TA.0b013e31815078ae}{Preventable or
potentially preventable mortality at a mature trauma center}. The
Journal of Trauma. 2007 Dec;63(6):1338-1346; discussion 1346-1347. }

\leavevmode\vadjust pre{\hypertarget{ref-noauthor_atls_2018}{}}%
\CSLLeftMargin{5. }%
\CSLRightInline{{ATLS} advanced trauma life support {[}Internet{]}. 10th
ed. Vol. 2018. Chicago: American College of Surgeons; 2018 {[}cited 2024
Sep 22{]}. Available from:
\url{https://cirugia.facmed.unam.mx/wp-content/uploads/2018/07/Advanced-Trauma-Life-Support.pdf}}

\leavevmode\vadjust pre{\hypertarget{ref-el-menyar_review_2019}{}}%
\CSLLeftMargin{6. }%
\CSLRightInline{El-Menyar A, Mekkodathil A, Abdelrahman H, Latifi R,
Galwankar S, Al-Thani H, et al. Review of {Existing} {Scoring} {Systems}
for {Massive} {Blood} {Transfusion} in {Trauma} {Patients}: {Where} {Do}
{We} {Stand}? Shock {[}Internet{]}. 2019 Sep {[}cited 2024 Oct
24{]};52(3):288. Available from:
\url{https://journals.lww.com/shockjournal/fulltext/2019/09000/review_of_existing_scoring_systems_for_massive.2.aspx}}

\leavevmode\vadjust pre{\hypertarget{ref-mackersie_pitfalls_2010}{}}%
\CSLLeftMargin{7. }%
\CSLRightInline{Mackersie RC. Pitfalls in the {Evaluation} and
{Resuscitation} of the {Trauma} {Patient}. Emergency Medicine Clinics of
North America {[}Internet{]}. 2010 Feb {[}cited 2024 Sep
10{]};28(1):1--27. Available from:
\url{https://www.sciencedirect.com/science/article/pii/S0733862709001199}}

\leavevmode\vadjust pre{\hypertarget{ref-ivatury_patient_2008}{}}%
\CSLLeftMargin{8. }%
\CSLRightInline{Ivatury RR, Guilford K, Malhotra AK, Duane T, Aboutanos
M, Martin N. \href{https://doi.org/10.1097/TA.0b013e318163359d}{Patient
safety in trauma: Maximal impact management errors at a level {I} trauma
center}. The Journal of Trauma. 2008 Feb;64(2):265-270; discussion
270-272. }

\leavevmode\vadjust pre{\hypertarget{ref-world_health_organization_guidelines_2009}{}}%
\CSLLeftMargin{9. }%
\CSLRightInline{World Health Organization. Guidelines for trauma quality
improvement programmes. 2009 {[}cited 2024 Sep 20{]};104. Available
from: \url{https://iris.who.int/handle/10665/44061}}

\leavevmode\vadjust pre{\hypertarget{ref-juillard_establishing_2009}{}}%
\CSLLeftMargin{10. }%
\CSLRightInline{Juillard CJ, Mock C, Goosen J, Joshipura M, Civil I.
\href{https://doi.org/10.1007/s00268-009-9959-8}{Establishing the
evidence base for trauma quality improvement: A collaborative
{WHO}-{IATSIC} review}. World Journal of Surgery. 2009
May;33(5):1075--86. }

\leavevmode\vadjust pre{\hypertarget{ref-hashmi_hospital-based_2013}{}}%
\CSLLeftMargin{11. }%
\CSLRightInline{Hashmi ZG, Haider AH, Zafar SN, Kisat M, Moosa A,
Siddiqui F, et al.
\href{https://doi.org/10.1097/TA.0b013e31829880a0}{Hospital-based trauma
quality improvement initiatives: First step toward improving trauma
outcomes in the developing world}. The Journal of Trauma and Acute Care
Surgery. 2013 Jul;75(1):60--68; discussion 68. }

\leavevmode\vadjust pre{\hypertarget{ref-donabedian_quality_1988}{}}%
\CSLLeftMargin{12. }%
\CSLRightInline{Donabedian A. The {Quality} of {Care}: {How} {Can} {It}
{Be} {Assessed}? JAMA {[}Internet{]}. 1988 Sep {[}cited 2024 Sep
19{]};260(12):1743--8. Available from:
\url{https://doi.org/10.1001/jama.1988.03410120089033}}

\leavevmode\vadjust pre{\hypertarget{ref-vang_shock_2022}{}}%
\CSLLeftMargin{13. }%
\CSLRightInline{Vang M, Østberg M, Steinmetz J, Rasmussen LS.
\href{https://doi.org/10.1007/s00068-022-01932-z}{Shock index as a
predictor for mortality in trauma patients: A systematic review and
meta-analysis}. European Journal of Trauma and Emergency Surgery:
Official Publication of the European Trauma Society. 2022
Aug;48(4):2559--66. }

\leavevmode\vadjust pre{\hypertarget{ref-tisherman_detailed_2015}{}}%
\CSLLeftMargin{14. }%
\CSLRightInline{Tisherman SA, Schmicker RH, Brasel KJ, Bulger EM, Kerby
JD, Minei JP, et al.
\href{https://doi.org/10.1097/SLA.0000000000000837}{Detailed description
of all deaths in both the shock and traumatic brain injury hypertonic
saline trials of the {Resuscitation} {Outcomes} {Consortium}}. Annals of
Surgery. 2015 Mar;261(3):586--90. }

\leavevmode\vadjust pre{\hypertarget{ref-eastridge_hypotension_2007}{}}%
\CSLLeftMargin{15. }%
\CSLRightInline{Eastridge BJ, Salinas J, McManus JG, Blackburn L, Bugler
EM, Cooke WH, et al.
\href{https://doi.org/10.1097/TA.0b013e31809ed924}{Hypotension begins at
110 mm {Hg}: Redefining "hypotension" with data}. The Journal of Trauma.
2007 Aug;63(2):291-297; discussion 297-299. }

\leavevmode\vadjust pre{\hypertarget{ref-oyetunji_redefining_2011}{}}%
\CSLLeftMargin{16. }%
\CSLRightInline{Oyetunji TA, Chang DC, Crompton JG, Greene WR, Efron DT,
Haut ER, et al.
\href{https://doi.org/10.1001/archsurg.2011.154}{Redefining hypotension
in the elderly: Normotension is not reassuring}. Archives of Surgery
(Chicago, Ill: 1960). 2011 Jul;146(7):865--9. }

\leavevmode\vadjust pre{\hypertarget{ref-mutschler_critical_2013}{}}%
\CSLLeftMargin{17. }%
\CSLRightInline{Mutschler M, Nienaber U, Brockamp T, Wafaisade A, Wyen
H, Peiniger S, et al. A critical reappraisal of the {ATLS}
classification of hypovolaemic shock: {Does} it really reflect clinical
reality? Resuscitation {[}Internet{]}. 2013 Mar {[}cited 2024 Oct
24{]};84(3):309--13. Available from:
\url{https://www.sciencedirect.com/science/article/pii/S0300957212003693}}

\leavevmode\vadjust pre{\hypertarget{ref-mutschler_renaissance_2013}{}}%
\CSLLeftMargin{18. }%
\CSLRightInline{Mutschler M, Nienaber U, Brockamp T, Wafaisade A, Fabian
T, Paffrath T, et al. Renaissance of base deficit for the initial
assessment of trauma patients: A base deficit-based classification for
hypovolemic shock developed on data from 16,305 patients derived from
the {TraumaRegister} {DGU}®. Critical Care {[}Internet{]}. 2013 Mar
{[}cited 2024 Oct 24{]};17(2):R42. Available from:
\url{https://pmc.ncbi.nlm.nih.gov/articles/PMC3672480/}}

\end{CSLReferences}

\end{document}
